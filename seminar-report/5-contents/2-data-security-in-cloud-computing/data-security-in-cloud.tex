\chapter{Data Security, its Issues and Various Methods of Data Security}

\paragraph{\hspace{24pt}}
Data security is one of the key concerns with Cloud Computing. Data security has three major factors to be considered. Security refers to the availability, integrity and confidentiality of data. Not providing or guaranteeing this may pose major issues for cloud vendors.\\\\

\textbf{Availability} {It is one of the most problematic issue faced by the consumers. Most cloud vendors have experienced downtime of their services which have affected majority of the users of cloud services. For example, Amazon servers have faced such issues which have been said to be Denial of Service attack. A cloud server must be available at all times and if not, a particular time period should be given as notice for the services to resume.}\\

\textbf{Integrity} {This is another important feature as cloud vendor or provider must provide. Data integrity refers to the accuracy and consistency of data stored on the cloud or any database as a matter of fact. Thin clients can be used to maintain integrity of data or security of data on the client side. This is possible since thin clients use as few resources as possible and they do no store any data. By doing this, personal information such as passwords cannot be stolen.}\\

\textbf{Confidentiality} {Personal or confidential data stored on the cloud should not be accessible by anyone other than the authorised entity. For example, data from parent company X which has been stored in child company Y should not be accessible by the employees of company Y, since it is confidential data of company X.}

\paragraph{\hspace{24pt}}
These are the factors essential for Data Security in any kind of network platform. Guaranteeing the mentioned factors ensures the security of the consumers or customers data.

\paragraph{\hspace{24pt}}
There are various issues that require addressing before an organization or enterprise considers switching to the cloud computing model. These issues are Privileged User Access, Data Location, Recovery, Long- term Viability, Data Segregation and Regulatory Compliance.

\paragraph{\hspace{24pt}}
The below mentioned graph shows the issues or concerns and its corresponding level/ percentage to which the consumers are affected.

\paragraph{\hspace{24pt}}
There are various algorithms and measures we can use to secure data on the virtualized environment of the cloud. They are Cryptography, Homomorphic Encryption, Diffie Hellman Algorithm, Rivest-Shamir-Adleman Algorithm(RSA), Container Clustering using Dockers.

\paragraph{\hspace{24pt}}
Cryptography can be used to encrypt the data stored on the cloud. But encrypting a large amount of data can be very challenging and time consuming.

\paragraph{\hspace{24pt}}
Other Encryption techniques such as Homomorphic Encryption and various algorithms such as RSA and DH algorithms can also be used to do the same. The major issue in using them is the excessive use of resources which eventually lead to an increased cost and time.

\paragraph{\hspace{24pt}}
Out of the five methods, Container Clustering using Dockers is one of the better and more efficient options to use for securing data on the cloud.

\paragraph{\hspace{24pt}}
Using dockers has its own advantage and because of this it beats all the other methods of securing data.